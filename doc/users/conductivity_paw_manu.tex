\documentclass[aps,prl,twocolumn,superscriptaddress]{revtex4}
\usepackage{graphicx}% Include figure files

\begin{document}
%
    \title{Conductivity calculations using the PAW formalism }


   \author{S. Mazevet, V. Recoules, M. Torrent, et F. Jollet}

   \affiliation{
D\'epartement de Physique Th\'eorique et Appliqu\'ee\\
CEA/DAM \^Ile-de-France\\
BP12, 91680 Bruy\`eres-le-Ch\^atel Cedex, France}         

   \date{\today}

To perform a conductivity calculation within the PAW formalism you need to first use 
a PAW potential and run a ground state calculation with the prtnabla variable set to 1 and prtwfk=1.

This calculates the necessary matrix elements and creates a file named filename\_OPT.

The postprocessor conducti read the file filename.OPT and calculate the electrical and thermal conductivity.

conducti <filename.files

where filename.files contains the input and output filenames.

filename.in contains the following variables in the PAW case:
2                 ! 2 for PAW calculations
filename     ! generic name of the ground state data files obtained with   prtwfk=1
0.0036749         !temperature 
1.000             ! K points weight
0.073119 0.0000001 5.00 1000   !gaussian width, omega\_min, omega\_max, nbr pts

Warning the conducti input file is for the moment different when used in the 
PAW and NC modes. In the NC the input file is (see /doc/users/conducti\_manuel.tex)
1                 ! 1 for norm-conserving calculations
t78o\_DS3\_1WF4 ! 1st DDK file
t78o\_DS4\_1WF5 ! 2nd DDK file
t78o\_DS5\_1WF6 ! 3rd DDK file
t78o\_DS2\_WFK  ! ground state data file obtained with   prtwfk=1
9.50049E-04   ! temperature
1.000         ! k point weigth
0.00735  2.0  ! Gaussian and frequency width; omega-max

\end{document}
